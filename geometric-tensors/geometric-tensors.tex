\documentclass[british]{beamer}
\usepackage{pgfpages}
%\setbeameroption{show notes on second screen}

\usepackage{lmodern}
\usepackage{fouriernc}
\usepackage[T1]{fontenc}
\usepackage[utf8]{inputenc}
\usepackage{graphicx}
\usepackage{csquotes}
\usepackage{babel}
\usepackage{braket}
\usepackage{nicefrac}

\useoutertheme{shadow}
% make a cambridge colour theme
\usetheme{Singapore}
\usefonttheme[onlymath]{serif}
\usefonttheme{structuresmallcapsserif}
\usefonttheme[onlysmall]{structurebold}

%\setbeamercovered{transparent}
\mode<presentation>

\title[Quantum Geometric Tensors]{L.C. Venuti and P. Zanardi \\ \vspace{0.2cm}
{\it Quantum Critical Scaling of the \\ Geometric Tensors}\\
 Phys. Rev. Lett. 99, 095701 (2007)}
\author{Gen Zhang}
\date{TCM Journal Club \\ Friday 5th November, 2010}

\begin{document}

\frame{\titlepage}

%\section[Outline]{}
%\frame{\tableofcontents}

\frame{
  \frametitle{Quantum Critical Points}
  \begin{itemize}
    \item Quantum critical points are critical transitions which occur at zero temperature, as some set of control parameters are adjusted
    \item As usual, can be studied by Ginzburg-Landau approach, i.e. order parameters, symmetry breaking, etc.
    \item Use quantum fidelity instead --- idea stolen from quantum information
  \end{itemize}
}

\frame{
  \frametitle{Roughly speaking...}
  \begin{itemize}
    \item \emph{``The strategy [...] is differential-geometric and information-theoretic in nature''}
    \item Compare (evaluate overlap of) ground states which are infinitesimally close
    \item Intuitively, at a phase transition, properties change dramatically/singularly with small shifts in parameters
    \item \emph{``besides its conceptual appeal, can have some practical relevance''}
  \end{itemize}
}

\frame{
  \frametitle{Geometry in Quantum Mechanics}
  \begin{itemize}
    \item Every quantum system has a projective Hilbert space of states: $P\mathcal{H}$, which contains \emph{rays} of normalised states
    \item Comes with a natural inner product on normalised states: $\Braket{\psi|\phi}$
    \item Extend to a natural, homogeneous metric: $$\gamma(\psi,\phi) = \arccos\left|\Braket{\psi|\phi}\right|$$
    \item Infinitesimally (acting on tangent vectors): $$g_\psi(u,v) = \Braket{u|\left(1-\Ket{\psi}\Bra{\psi}\right)v}$$
  \end{itemize}
}

\frame{
  \frametitle{Geometry of parameter space}
  \begin{itemize}
    \item Assume parameters can be made into a manifold $\mathcal{M}$
    \item At each point $\lambda$ of $\mathcal{M}$, there is a Hamiltonian $$H(\lambda)=\sum_{n=0}^{\mathrm{dim} \mathcal{H}} E_n \Ket{\psi_n}\Bra{\psi_n},\textrm{ with }E_{n+1} \ge E_n$$
    \item Assume $\lambda \mapsto H(\lambda)$ is smooth
    \item Groundstate gives map: $\lambda \mapsto \Ket{\psi_0}$
    \item Pullback $g_\psi(u,v)$ to $\mathcal{M}$: $$Q_{\mu\nu}(\lambda) = \Braket{\partial_\mu \psi_0|\partial_\nu \psi_0} - \Braket{\partial_\mu \psi_0|\psi_0}\Braket{\psi_0|\partial_\nu \psi_0}$$
  \end{itemize}
}

\frame{
  \frametitle{Quantum geometric tensor (real)}
  \begin{itemize}
    \item All physical information resides in $Q_{\mu\nu}$
    \item In general, $Q_{\mu\nu}$ is complex --- not a metric
    \item But it is Hermitian --- so real part $g_{\mu\nu}$ is a proper metric
    \item Quantum fidelity, aka overlap: $\mathcal F(\psi,\phi) = \left|\Braket{\psi|\phi}\right|$
    \item $g_{\mu\nu}$ is leading order term for infinitesimal displacements
    \item In the thermodynamic limit, $g_{\mu\nu}$ develops singularities
  \end{itemize}
}

\frame{
  \frametitle{Quantum geometric tensor (imaginary)}
  \begin{itemize}
    \item The imaginary part 
      \begin{align*}
        F_{\mu\nu}&= \Im \Braket{\partial_\mu \psi_0|\partial_\nu \psi_0} \\
                  &= \Braket{\partial_\mu \psi_0|\partial_\nu \psi_0} - \Braket{\partial_\nu \psi_0|\partial_\mu \psi_0} \\
                  &= \partial_\mu A_\nu - \partial_\nu A_\mu
      \end{align*}
    \item Antisymmetric 2-form --- symplectic structure on $\mathcal M$
    \item 1-form $A_\mu = \Braket{\psi_0|\partial_\mu \psi_0}$ is the Berry connection
    \item Contour integral of $A_\mu$ gives the Berry phase of adiabatic transport along that contour
    \item Again singularities develop in the thermodynamic limit --- creases at quantum critical points
  \end{itemize}
}

\frame{
  \frametitle{Physical intuition}
  \begin{itemize}
    \item Differentiate $H(\lambda)\Ket{\psi_0} = E_0 \Ket{\psi}$: $$Q_{\mu\nu} = \sum_{n\neq0} \frac{\Braket{\psi_0|\partial_\mu H|\psi_0}\Braket{\psi_0|\partial_\nu H|\psi_0}}{\left(E_n-E_0\right)^2} $$
    \item At critical points level crossings occur --- $\epsilon_n = E_n-E_0$ vanishes in the thermodynamic limit
  \end{itemize}
}

\frame{
  \frametitle{Finite scaling}
  \begin{itemize}
    \item From spectral expansion above, get inequality:
      \begin{align*}
        \left|Q_{\mu\nu}\right| &\le \sum_{n>0}\epsilon_n^{-2} \left|\Braket{\psi_0|\delta H|\psi_0}\right|^2 
          < \epsilon_1^{-2} \sum_{n>0} \left|\Braket{\psi_0|\delta H|\psi_0}\right|^2 \\
          &= \epsilon_1^{-2} \left( \Braket{\delta H \delta H^\dagger} - \left| \Braket{\delta H} \right|^2\right)
      \end{align*}
    \item Assuming $\delta H$ is local, the averages above goes as $L^d K$, where $K$ is finite and independent of system size
    \item Then, $$\lim_{L\rightarrow\infty} \frac{|Q_{\mu\nu}|}{L^d} < K \epsilon_1^2,$$ which is finite
    \item \emph{``Super-extensive behaviour of any of the components of $Q$ for system with local interaction implies a vanishing gap in the thermodynamic limit''}
  \end{itemize}
}

\frame{
  \frametitle{Boring bits}
  \begin{itemize}
    \item $Q_{\mu\nu}$ can be given an integral representation in terms of correlation functions
    \item Allows scaling analysis
    \item Usual assumptions yields usual results for critical exponents
    \item Explicit calculation with $XXZ$ spin-$\nicefrac{1}{2}$ chain
  \end{itemize}
}

\frame{
  \frametitle{Take home message}
  \begin{center}
    \emph{``The main message of this paper is that apparently unrelated results can be understood in unified fashion by unveiling the underlying common differential-geometric structure and analysing its quantum-critical behaviour''}
  \end{center}
}

\end{document}
