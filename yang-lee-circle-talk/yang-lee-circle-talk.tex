\documentclass[british]{beamer}
\usepackage{pgfpages}
%\setbeameroption{show notes on second screen}

\usepackage{lmodern}
\usepackage{fouriernc}
\usepackage[T1]{fontenc}
\usepackage[utf8]{inputenc}
\usepackage{graphicx}
\usepackage{csquotes}
\usepackage{babel}

\useoutertheme{shadow}
% make a cambridge colour theme
% \usecolortheme{beetle}
\usefonttheme[onlymath]{serif}
\usefonttheme{structuresmallcapsserif}
\usefonttheme[onlysmall]{structurebold}

%\setbeamercovered{transparent}
\mode<presentation>

\title[Yang-Lee Circle Theorem]{C. N. Yang and T. D. Lee \\ \vspace{0.2cm}
{\it Statistical Theory of Equations of State and Phase Transitions}, Phys. Rev. 87 (1952)\\ \vspace{0.15cm}
{\it I. Theory of Condensation} \&\\
{\it II. Lattice Gas and Ising Model}}
\author{Gen Zhang}
\date{TCM Journal Club \\ Friday 12th February, 2010}

\begin{document}

\frame{\titlepage}

%\section[Outline]{}
%\frame{\tableofcontents}

\section{Context and General Theory}
\subsection{In Fair Verona...}
\frame
{
  \frametitle{Theories of Condensation}

  \begin{itemize}
  \item Not clear that statistical mechanics can give equations of state for both gas and liquid from the same interactions: ``{\it How can the gas molecules know when they have to coagulate to form a liquid or solid?}''\footnote{M. Born and K. Fuchs, {\it The statistical mechanics of condensing systems}, Proc. Roy. Soc. (London) A166, 391 (1938)}
  \item Mayer's theory\footnote{J. E. Mayer, {\it The statistical mechanics of condensing systems}, J. Chem. Phys. 5, 67 (1937)} of expanding the pressure $p$ at low chemical potential gives unphysical results --- no liquid state at all at finite density
  \item In general how to get macroscopic quantities, e.g. isotherms, effectively from statistical mechanics
  \end{itemize}
  \note{mention: elementary methods vs ``big machine''}
}

\subsection{The Partition Function}
\frame
{
  \frametitle{The Model}

  \begin{itemize}
  \item Classical molecules --- but still indistinguishable.
  \item Finite range interactions --- dimensionality still makes sense
  \item Hard core repulsion --- disallow infinite density
  \item No infinitely deep potential wells --- disallow perfect crystallisation
  \end{itemize}
}

\frame
{
  \frametitle{The Partition Function}

  Consider then a box of volume $V$ kept at a temperature $T$ and chemical potential $\mu$ by some reservoir. The relative probability of having $n$ atoms in the box is $$Q_n y^n/n!$$ where $$Q_n = \idotsint_V dx_1 \ldots dx_n \exp(-U/kT)$$ is the configurational part of the partition function and $$y = \left(\frac{2\pi m kT}{h^2}\right)^{\frac{3}{2}} \exp(\mu/kT).$$
}

\frame
{
  \frametitle{The Partition Function}
  
  The grand partition function is then quite straightforwardly $$\mathcal{Q}_V = \sum_{n=0}^N \frac{1}{n!} Q_n y^n,$$ where $N$ is the maximum number of molecules possible in $V$. Note that $\mathcal{Q}_V$ is:

  \begin{itemize}
  \item<1-> a finite degree polynomial
  \item<2-> factorisable into (up to a constant) $$\mathcal{Q}_V = \prod_{i=1}^N\left(1-\frac{y}{y_i}\right)$$ 
  \note<2->{Write partition function $\mathcal{Q}_V$ factorised, on board}
  \item<3-> a (complex) analytic function of the fugacity $y$
  \end{itemize}
}

\frame
{
  \frametitle{Thermodynamic relations}
  
  The thermodynamic limit is given by $V \rightarrow \infty$. Accordingly, the presure and density are given by
  \begin{align*}
  \frac{p}{kT} &= \lim_{V \rightarrow \infty}\frac{1}{V}\ln \mathcal{Q}_V, \\
  \rho &= \lim_{V \rightarrow \infty}\frac{\partial}{\partial \ln y}\frac{1}{V}\ln \mathcal{Q}_V.
  \end{align*}
  Note that $\rho$ involves a double limit process. This will need to be handled with care.
  \note{talk about how zeros of partition function turn into poles of density, and their motion in the complex plane}
}

\subsection{Analytic Properties and Phases}
\frame
{
  \frametitle{Pressure is Well-defined}

  \begin{theorem}
  For all positive real $y$, $V^{-1} \ln \mathcal{Q}_V$ approaches, as $V\rightarrow \infty$, a limit which is independent of the shape of $V$. Furthermore, this limit is a continuous, monotonically increasing function of $y$.
  \end{theorem}
}

\frame
{
  \frametitle{Density is Well-defined}

  \begin{theorem}<1->
  If a region $R$ of the complex $y$ plane, containing a segment of the positive real axis, is always free of roots, then in this region as $V\rightarrow \infty$ the quantities: $$\frac{1}{V}\ln \mathcal{Q}_V, \left(\frac{\partial}{\partial \ln y}\right)\frac{1}{V}\ln \mathcal{Q}_V, \left(\frac{\partial}{\partial \ln y}\right)^2\frac{1}{V}\ln \mathcal{Q}_V, \ldots$$ all approach limits which are analytic with respect to $y$.
  \end{theorem}
  
  In addition:
  \begin{itemize}
  \item<2-> The operations of $\partial/\partial \ln y$ and $\lim_{V\rightarrow \infty}$ commute, so that within $R$ $$\rho = \frac{\partial}{\partial \ln y}\left(\frac{p}{kT}\right).$$
  \item<3-> $\rho$ is also an increasing function of $y$.
  \end{itemize}

  \note{draw diagrams for a given set of roots; specific volume; talk about phase transitions; compare with Mayer's theory}
}

\section{Ferromagnetic Ising Model}
\subsection{Lattice Gas and Ising Model}
\frame
{
  \frametitle{Lattice Gas Equivalence}

  The Ising model is equivalent to a lattice gas, where we identify:
  \begin{align*}
  \textrm{total spins, $N$} &= \textrm{volume} \\
  \textrm{$\downarrow$ spins} &= \textrm{atoms} \\
  2/(1-m) &= \textrm{specific volume $v$} \\
  -f-B &= \textrm{pressure $p$} \\
  s \exp(-2B/kT) &= \textrm{fugacity $y$}
  \end{align*}
}

\subsection{Distribution of Roots}
\frame
{
  \frametitle{Distribution of Roots}

  The partition function for $N$ spins is $$\mathcal{P}_N = \sum P_n z^n$$ where $z = \exp(-2B/kT)$, and $P_n$ is the Boltzmann factor for having $n$ $\downarrow$ spins.

  Then note:
  \begin{itemize}
  \item<2-> Symmetry under $n \leftrightarrow N-n$
  \note<2->{inversion symmetry of roots; }
  \item<3-> Special structure of $P_n$
  \note<3->{$P_n$ are sums of products}
  \end{itemize}

}

\frame
{
  \frametitle{``Theorem 3''}

  Let $x_{\alpha \beta}=x_{\beta \alpha}$ ($\alpha \neq \beta; \alpha, \beta = 1,2,\ldots n$) be real numbers whose absolute values are less than or equal to 1. Divide the integers $1,2,\ldots n$ into two groups $a$ and $b$ so that there are $\gamma$ integers in group $a$ and $(n-\gamma)$ in group b. Define 
  \begin{align*}
  P_\gamma &= \sum \frac{1}{\gamma! (n-\gamma)!} \prod_{i=1}^{\gamma} \prod_{j=1}^{n-\gamma} x_{{a_i} {b_j}}, \\
  \mathcal{P}(z) &= 1+P_1 z + P_2 z^2 + \ldots + P_{n-1} z^{n-1} + z^n.
  \end{align*}
  Then the roots of the equation $$\mathcal{P}=0$$ are on the unit circle.

  \note{mention: distribution of roots of random polynomials tend to avoid the unit circle, and ``neat'' points esp.}
}

\subsection{Thermodynamics}
\frame
{
  \frametitle{Physical Consequences}

  \begin{itemize}
  \item<1-> There is at most one phase transition in the ferromagnetic Ising model, occuring at $z=1$. The $m$-$B$ curve is thus continuous everywhere, except possibly at $B=0$.
  \note<1->{Ising in 1D has no phase transition --- no pinch}
  \item<2-> Similarly, the lattice gas has (possibly) a phase transition at $y=s$. 
  \item<3-> The thermodynamic properties are entirely contained by the root distribution function $g(\theta)$ where $N g(\theta) d\theta$ is the number of roots with $z$ between $e^{i\theta}$ and $e^{i(\theta+d\theta)}$.
  \end{itemize}
}

\frame
{
  \frametitle{Phase Transitions}

  \begin{itemize}
  \item The actual phase transitions depend on the behaviour of $g(\theta)$ near $\theta = 0$. So write $$g(\theta) = g(0) + a\left|\theta\right|^n+\ldots.$$
  \item The $g(0)$ term, if non-vanishing, gives a first-order transition.
  \item Otherwise $n$ controls the order of the transition.
  \end{itemize}
}

\end{document}
